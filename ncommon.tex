\usepackage{amsmath}
\usepackage{amssymb}
\usepackage{amsfonts}
\usepackage{textcomp}
\usepackage{mathtools}
\usepackage{cancel}
\usepackage{gensymb}
\usepackage{esint}
\usepackage{esvect}
\usepackage{braket}
\usepackage{siunitx}
\usepackage{pifont}
%\usepackage{MnSymbol}
\usepackage{mathrsfs}
\usepackage{mhchem}
\usepackage{bm}
\usepackage{tikz}
\usepackage[american, siunitx, cuteinductors]{circuitikz}
\usepackage{xcolor}
\usepackage{tcolorbox}
\usepackage{diagbox}
\usepackage{ulem}
\usepackage{accents}
\usepackage{clrscode3e}
\usepackage{wasysym}
\usepackage{sansmath}
\usepackage{subcaption}

\let\savedegree\degree
\let\degree\relax
%\usepackage{mathabx}
\let\degree\savedegree

\let\oldemptyset\emptyset
\let\emptyset\varnothing

\normalem

\renewcommand{\gets}{\leftarrow}
\newcommand{\notmodels}{\not\models}
\newcommand{\rmodels}{\ensuremath{\mathrel{=\!\!\!|}}}
\newcommand{\on}{\operatorname}

% Thanks stack exchange
\DeclareMathOperator*{\sumint}{%
\mathchoice%
  {\ooalign{$\displaystyle\sum$\cr\hidewidth$\displaystyle\int$\hidewidth\cr}}
  {\ooalign{\raisebox{.14\height}{\scalebox{.7}{$\textstyle\sum$}}\cr\hidewidth$\textstyle\int$\hidewidth\cr}}
  {\ooalign{\raisebox{.2\height}{\scalebox{.6}{$\scriptstyle\sum$}}\cr$\scriptstyle\int$\cr}}
  {\ooalign{\raisebox{.2\height}{\scalebox{.6}{$\scriptstyle\sum$}}\cr$\scriptstyle\int$\cr}}
}

\DeclarePairedDelimiterX{\infdivx}[2]{(}{)}{%
  #1\;\delimsize\|\;#2%
}
\newcommand{\infdiv}{D\infdivx}

\newcommand\independent{\protect\mathpalette{\protect\independenT}{\perp}}
\def\independenT#1#2{\mathrel{\rlap{$#1#2$}\mkern2mu{#1#2}}}
\newcommand{\indep}{\independent}

\newcommand{\expectation}[2]{\mathbb E_{#1}\left[#2\right]}

\DeclareMathOperator*{\LessGtr}{\lessgtr}
\DeclareMathOperator*{\GtrLess}{\gtrless}

\newcommand\com{\text{%
    \tikz[radius=0.3em] {%
        \fill (0,0) -- ++(0.3em,0) arc [start angle=0,end angle=90] -- ++(0,-0.6em) arc [start angle=270, end angle=180];%
        \draw (0,0) circle;%
    }%
}}
\newcommand\scom{\text{%
    \tikz[radius=0.2em] {%
        \fill (0,0) -- ++(0.2em,0) arc [start angle=0,end angle=90] -- ++(0,-0.4em) arc [start angle=270, end angle=180];%
        \draw (0,0) circle;%
    }%
}}

\renewcommand{\Re}{\operatorname{Re}}
\renewcommand{\Im}{\operatorname{Im}}

\ctikzset{
	bipoles/length=1.0cm,
	bipoles/vsourceam/inner plus={\fontsize{5}{7}\selectfont $+$},
	bipoles/vsourceam/inner minus={\fontsize{5}{7}\selectfont $-$},
	logic ports=ieee,
	logic ports/scale=0.8
}

\everymath{\displaystyle} % Always use display style math, even when inline

\newcommand{\msout}[1]{\text{\sout{\ensuremath{#1}}}}

\mathchardef\arr="017E % character 7E from textfont 1 is the vector arrow 
\newcommand\undervec[1]{\underaccent{\vec}{#1}}
\newcommand{\uvec}{\undervec}

\newcommand{\vectrix}[1]{\undervec{\bm{\mathcal{#1}}}}
\newcommand{\vcx}{\vectrix}

\newcommand{\dota}{{\mbox{\large $\cdot$}}}
\newcommand{\ddota}{{\dota\dota}}
\newcommand{\dotb}{{\mbox{\tiny $\circ$}}}
\newcommand{\ddotb}{{\dotb\dotb}}
\newcommand{\dotc}{{\mbox{\tiny $\triangle$}}}
\newcommand{\ddotc}{{\dotc\dotc}}

\newcommand{\tarc}{\mbox{\large$\frown$}}
\newcommand{\overarc}[2][-3ex]{{#2}{\kern #1{\raisebox{1.5ex}{\tarc}}}}

\newcommand{\sans}[1]{\begin{sansmath}\text{${#1}$}\end{sansmath}}

\newcommand{\ssquare}{{\scalebox{0.5}{$\square$}}}
\newcommand{\striangle}{{\scalebox{0.5}{$\triangle$}}}

% Increase maximum list nesting depth
\usepackage{enumitem}
\renewlist{itemize}{itemize}{9}
\setlist[itemize,1]{label=\textbullet}
\setlist[itemize,2]{label=--}
\setlist[itemize,3]{label=*}
\setlist[itemize,4]{label=\textbullet}
\setlist[itemize,5]{label=--}
\setlist[itemize,6]{label=*}
\setlist[itemize,7]{label=\textbullet}
\setlist[itemize,8]{label=--}
\setlist[itemize,9]{label=*}

\newcommand{\cmark}{\ding{51}}
\newcommand{\xmark}{\ding{55}}

\newcommand{\dbar}{d\hspace*{-0.08em}\bar{}\hspace*{0.1em}}
\newcommand{\ddbar}{\mathrm{d}\hspace*{-0.08em}\bar{}\hspace*{0.1em}}
\newcommand{\grad}{\vec\nabla}
\newcommand{\del}{\vec\nabla}
\newcommand{\udel}{\uvec\nabla}
\newcommand{\bv}[1]{\vec{\bm{#1}}}
\newcommand{\dotvec}[1]{\dot{\vec{#1}}}

\newcommand{\equal}{=} % Because tikz does not like plain equal signs in labels

\newcommand{\xor}{\oplus}

\newtcolorbox{note-note}{title=Note,colback=blue!5!white,colframe=blue,fonttitle=\bfseries}
\newtcolorbox{note-important}{title=Important,colback=red!5!white,colframe=red!75!black,fonttitle=\bfseries}
\newtcolorbox{note-theorem}{title=Theorem,colback=green!5!white,colframe=green!75!black,fonttitle=\bfseries}
\newtcolorbox{note-definition}{title=Definition,colback=green!5!white,colframe=green!75!black,fonttitle=\bfseries}
\newtcolorbox{note-equation}{title=Equation,colback=green!5!white,colframe=green!75!black,fonttitle=\bfseries}
\newtcolorbox{note-summary}{title=Summary,colback=yellow!10!white,colframe=yellow!75!black,fonttitle=\bfseries}
\newcommand{\noteNote}[1]{\begin{note-note}#1\end{note-note}}
\newcommand{\noteImportant}[1]{\begin{note-important}#1\end{note-important}}
\newcommand{\noteThm}[1]{\begin{note-theorem}#1\end{note-theorem}}
\newcommand{\noteDefn}[1]{\begin{note-definition}#1\end{note-definition}}
\newcommand{\noteEqn}[1]{\begin{note-equation}#1\end{note-equation}}
\newcommand{\noteSummary}[1]{\begin{note-summary}#1\end{note-summary}}

\DeclareMathOperator{\rank}{rank} % Rank of a matrix
\DeclareMathOperator{\tr}{tr} % Trace of a matrix
\DeclareMathOperator{\diag}{diag} % Diagonal matrix
\DeclareMathOperator{\im}{im} % Image
\DeclareMathOperator{\proj}{proj} % Projection
\DeclareMathOperator{\Span}{span} % Note \span is already defined
\DeclareMathOperator{\Null}{null}
\DeclareMathOperator{\row}{row}
\DeclareMathOperator{\col}{col}
\DeclareMathOperator{\curl}{curl} % Curl of vector field
\DeclareMathOperator{\divergence}{div} % Divergence of vector field
\DeclareMathOperator{\erf}{erf}
% Hyperbolic trig functions
\DeclareMathOperator{\sech}{sech}
\DeclareMathOperator{\csch}{csch}
\DeclareMathOperator{\Coth}{coth} % note coth is already defined
\DeclareMathOperator{\adj}{adj}
% div for divergence is defined as \divergence because \div is already used for the division symbol; uncomment the code below to redefine it
%\let\divsymb\div
%\let\div\relax
%\DeclareMathOperator{\div}{div}
\DeclareMathOperator{\erfc}{erfc} % complementary error function
\DeclareMathOperator{\var}{var}
\DeclareMathOperator{\Var}{Var}
\DeclareMathOperator{\cov}{cov}
\DeclareMathOperator{\Cov}{Cov}

\DeclareMathOperator{\argmin}{argmin}
\DeclareMathOperator{\argmax}{argmax}

\DeclareMathOperator{\sign}{sign}
\DeclareMathOperator{\sgn}{sgn}

\let\argmax\relax
\let\argmin\relax
\DeclareMathOperator*{\argmax}{argmax}
\DeclareMathOperator*{\argmin}{argmin}

\DeclarePairedDelimiter{\abs}{\lvert}{\rvert} % Proper absolute values etc
\DeclarePairedDelimiter{\norm}{\lVert}{\rVert}
\DeclarePairedDelimiter{\ceil}{\lceil}{\rceil}
\DeclarePairedDelimiter{\floor}{\lfloor}{\rfloor}

\newcommand{\diff}[2]{\frac{\mathrm{d}#1}{\mathrm{d}#2}} % Derivative, \diff{func}{var}
\newcommand{\diffn}[3]{\frac{\mathrm{d}^{#1}#2}{\mathrm{d}#3^{#1}}} % Multi-derivative, \diffn{power}{func}{var}
\newcommand{\pdiff}[2]{\frac{\partial#1}{\partial#2}} % Partial derivative, \pdiff{func}{var}
\newcommand{\pdiffc}[3]{\left(\pdiff{#1}{#2}\right)_{#3}} % Partial derivative keeping a variable constant, \pdiffc{func}{var}{var}
\newcommand{\pdiffn}[3]{\frac{\partial^{#1}#2}{\partial{#3}^{#1}}} % Partial multi-derivative, \pdiffn{power}{func}{var}
\newcommand{\ppdiff}[3]{\frac{\partial^2 #1}{\partial #2 \partial #3}}
\newcommand{\pppdiff}[4]{\frac{\partial^3 #1}{\partial #2 \partial #3 \partial #4}}
\newcommand{\Diff}[2]{\frac{\mathrm{D}#1}{\mathrm{D}#2}}
\newcommand{\laplace}[1]{\mathcal{L}\left\{ #1 \right\}} % Laplace transform
\newcommand{\ilaplace}[1]{\mathcal{L}^{-1}\left\{ #1 \right\}} % Inverse Laplace transform
\newcommand{\fourier}[1]{\mathcal{F}\left\{ #1 \right\}}
\newcommand{\ifourier}[1]{\mathcal{F}^{-1}\left\{ #1 \right\}}

% Derivative and other things evaluated at
\newcommand{\eval}[2]{\left.#1\right\rvert_{#2}}
\renewcommand{\inteval}[3]{\left[#1\right]_{#2}^{#3}}
\newcommand{\integral}[5][\int]{#1_{#2}^{#3}#4\,\mathrm{d}#5}

% Cross product because I always forget that \cross isn't a thing
\newcommand{\cross}{\times}

\newcommand{\intinf}{\int_{-\infty}^{\infty}}
\newcommand{\intuinf}[1]{\int_{#1}^{\infty}}
\newcommand{\intlinf}[1]{\int_{-\infty}^{#1}}

\newcommand{\reals}{\mathbb{R}}
\newcommand{\complex}{\mathbb{C}}
\newcommand{\integers}{\mathbb{Z}}
\newcommand{\naturals}{\mathbb{N}}
\newcommand{\rationals}{\mathbb{Q}}

\newcommand{\funcip}[2]{\left(#1\middle|#2\right)}

\newcommand{\matdim}[3][\mathbb{R}]{{}^{#2} #1 ^{#3}}

% Differential d
\newcommand{\dd}[1]{\mathrm{d}#1}
\newcommand{\da}{\dd a}
\newcommand{\db}{\dd b}
\newcommand{\dc}{\dd c}
\newcommand{\de}{\dd e}
\newcommand{\df}{\dd f}
\newcommand{\dm}{\dd m}
\newcommand{\dk}{\dd k}
\newcommand{\dr}{\dd r}
\newcommand{\ds}{\dd s}
\newcommand{\dt}{\dd t}
\newcommand{\du}{\dd u}
\newcommand{\dv}{\dd v}
\newcommand{\dw}{\dd w}
\newcommand{\dx}{\dd x}
\newcommand{\dy}{\dd y}
\newcommand{\dz}{\dd z}
\newcommand{\dA}{\dd A}
\newcommand{\dB}{\dd B}
\newcommand{\dC}{\dd C}
\newcommand{\dV}{\dd V}
\newcommand{\dR}{\dd R}
\newcommand{\dS}{\dd S}
\newcommand{\dT}{\dd T}
\newcommand{\dtheta}{\dd \theta}
\newcommand{\drho}{\dd \rho}
\newcommand{\dphi}{\dd \phi}
\newcommand{\dvarphi}{\dd \varphi}
\newcommand{\dtau}{\dd \tau}

\newcommand{\Vbar}{\msout{V}}
\newcommand{\ReN}{\text{Re}} % Reynolds number
\newcommand{\FoN}{\text{Fo}} % Fourier number
\newcommand{\BiN}{\text{Bi}} % Biot number
\newcommand{\NuN}{\text{Nu}} % Nusselt number
\newcommand{\PrN}{\text{Pr}} % Prandtl number
\newcommand{\FrN}{\text{Fr}} % Froude number

\newcommand{\union}{\cup}
\newcommand{\intersect}{\cap}
\newcommand{\intersection}{\cap}

% Basis vectors
\newcommand{\ihat}{\hat{\imath}}
\newcommand{\jhat}{\hat{\jmath}}
\newcommand{\khat}{\hat{k}}

% Shorthand for 2 or 3 piecewise
\newcommand{\twopiece}[2]{\begin{cases} {#1} \\ {#2} \end{cases}}
\newcommand{\threepiece}[3]{\begin{cases} {#1} \\ {#2} \\ {#3} \end{cases}}
\newcommand{\fourpiece}[4]{\begin{cases} {#1} \\ {#2} \\ {#3} \\ {#4} \end{cases}}
\newcommand{\fivepiece}[5]{\begin{cases} {#1} \\ {#2} \\ {#3} \\ {#4} \\ {#5} \end{cases}}
% Because for some reason you can't use & inside the macro
\newcommand{\twocond}[4]{\begin{cases} {#1} & {#2} \\ {#3} & {#4} \end{cases}}
\newcommand{\threecond}[6]{\begin{cases} {#1} & {#2} \\ {#3} & {#4} \\ {#5} & {#6} \end{cases}}
\newcommand{\fourcond}[8]{\begin{cases} {#1} & {#2} \\ {#3} & {#4} \\ {#5} & {#6} \\ {#7} & {#8} \end{cases}}

% Matrices
\newcommand{\mattwo}[5][b]{\begin{#1matrix} {#2} & {#3} \\ {#4} & {#5} \end{#1matrix}}
\newcommand{\mattwothree}[7][b]{\begin{#1matrix} {#2} & {#3} & {#4} \\ {#5} & {#6} & {#7} \end{#1matrix}}
\newcommand{\matthreetwo}[7][b]{\begin{#1matrix} {#2} & {#3} \\ {#4} & {#5} \\ {#6} & {#7} \end{#1matrix}}
% Because TeX can't take more than 9 arguments in a macro...
\newcommand{\matthree}[9]{\begin{matrix} {#1} & {#2} & {#3} \\ {#4} & {#5} & {#6} \\ {#7} & {#8} & {#9} \end{matrix}}
\newcommand{\matthreeb}[9]{\begin{bmatrix} {#1} & {#2} & {#3} \\ {#4} & {#5} & {#6} \\ {#7} & {#8} & {#9} \end{bmatrix}}
\newcommand{\matthreev}[9]{\begin{vmatrix} {#1} & {#2} & {#3} \\ {#4} & {#5} & {#6} \\ {#7} & {#8} & {#9} \end{vmatrix}}
\newcommand{\matthreep}[9]{\begin{pmatrix} {#1} & {#2} & {#3} \\ {#4} & {#5} & {#6} \\ {#7} & {#8} & {#9} \end{pmatrix}}
\newcommand{\matthreeB}[9]{\begin{Bmatrix} {#1} & {#2} & {#3} \\ {#4} & {#5} & {#6} \\ {#7} & {#8} & {#9} \end{Bmatrix}}

\newcommand{\diagthree}[4][b]{\begin{#1matrix} {#2} & & \\ & {#3} & \\ & & {#4} \end{#1matrix}}
\newcommand{\diagfour}[5][b]{\begin{#1matrix} {#2} & & & \\ & {#3} & & \\ & & {#4} & \\ & & & {#5} \end{#1matrix}}

\newenvironment{amatrix}[1]{%
  \left[\begin{array}{@{}*{#1}{c}|c@{}}
}{%
\end{array}\right]
}

\makeatletter % Column vector (variadic), \cvec[bracket_type]{x1}{x2}{x3}...
\newcommand{\cvec}[2][b]{\begin{#1matrix} #2 \cvecchecknextarg{#1}}
\newcommand{\cvecchecknextarg}[1]{\@ifnextchar\bgroup{\cvecnextarg{#1}}{\end{#1matrix}}}
\newcommand{\cvecnextarg}[2]{\\ #2 \@ifnextchar\bgroup{\cvecnextarg{#1}}{\end{#1matrix}}}
\makeatother

\makeatletter % Row vector (variadic), \rvec[bracket_type]{x1}{x2}{x3}...
\newcommand{\rvec}[2][b]{\begin{#1matrix} #2 \rvecchecknextarg{#1}}
\newcommand{\rvecchecknextarg}[1]{\@ifnextchar\bgroup{\rvecnextarg{#1}}{\end{#1matrix}}}
\newcommand{\rvecnextarg}[2]{& #2 \@ifnextchar\bgroup{\rvecnextarg{#1}}{\end{#1matrix}}}
\makeatother

\makeatletter
\newcommand{\alignedlines}[3][]{\begin{aligned}[#1] &\hphantom{{}#2{}} #3 \alignedlineschecknextarg{#2}}
\newcommand{\alignedlineschecknextarg}[1]{\@ifnextchar\bgroup{\alignedlinesnextarg{#1}}{\end{aligned}}}
\newcommand{\alignedlinesnextarg}[2]{\\ &#1 #2 \@ifnextchar\bgroup{\alignedlinesnextarg{#1}}{\end{aligned}}}

\newcommand{\alignedlinestwo}[4][]{\begin{aligned}[#1] #3 &#2 #4 \alignedlinestwochecknextarg{#2}}
\newcommand{\alignedlinestwochecknextarg}[1]{\@ifnextchar\bgroup{\alignedlinestwonextarg{#1}}{\end{aligned}}}
\newcommand{\alignedlinestwonextarg}[2]{\\ &#1 #2 \@ifnextchar\bgroup{\alignedlinestwonextarg{#1}}{\end{aligned}}}
\makeatother

\newcommand{\alignedeqn}[1][]{\alignedlines[#1]{=}}
\newcommand{\alignedimp}[1][]{\alignedlines[#1]{\implies}}
\newcommand{\alignedeqntwo}[1][]{\alignedlinestwo[#1]{=}}
\newcommand{\alignedimptwo}[1][]{\alignedlinestwo[#1]{\implies}}

\newcommand{\Tag}{& \qquad &}
\newcommand{\Ttag}[1]{\Tag \text{#1}}
\newcommand{\Tagcal}[1]{\Tag \mathcal{#1}}

